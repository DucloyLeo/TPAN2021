\documentclass[12pt]{article}
\usepackage[utf8]{inputenc}
\usepackage[T1]{fontenc}
\usepackage[a4paper,left=2cm,right=2cm,top=2cm,bottom=2cm]{geometry}
\usepackage[frenchb]{babel}
\usepackage{libertine}
\usepackage[pdftex]{graphicx}
\usepackage{amsmath}
\usepackage{amssymb}
\usepackage{mathtools}
% Pour tracer des courbes et divers graphiques, et utiliser des couleurs
\usepackage{pstricks}
\usepackage{pstricks-add}
\usepackage{hyperref} %Hyperlien pour les equations
\usepackage{listingsutf8}
\usepackage{color}
\usepackage{movie15} %mettre des gif (ne fonctionne pas)
\usepackage{wrapfig}


\makeatletter
\setlength{\parskip}{1ex plus 0.5ex minus 0.2ex}
\newcommand{\hsp}{\hspace{20pt}}
\newcommand{\HRule}{\rule{\linewidth}{0.5mm}}
\newlength{\boxed@align@width}
\newcommand{\boxedalign}[3][black]{
  \hspace*{\fboxsep}\hspace*{\fboxrule}% pour que le centrage soit le même que la formule avec \boxed
  \color{#1}#2
  &
  \setlength{\boxed@align@width}{%
     \widthof{$\displaystyle#2$}+\fboxsep+\fboxrule}%
  \hspace{-\boxed@align@width}%
  \addtolength{\boxed@align@width}{-\fboxsep-\fboxrule}%
  \begingroup
    \color{#1}\boxed{\vphantom{#2}\hspace{\boxed@align@width}#3}%
  \endgroup
}
\newcommand*\centermathcell[1]{\omit\hfil$\displaystyle#1$\hfil\ignorespaces}
% This is the default font in Spyder
\newcommand*{\pyfontfamily}{\fontfamily{DejaVuSansMono-TLF}\selectfont}
\makeatother

% These are close to the default font colors in Spyder
\definecolor{pycommentcol}{RGB}{153,153,153}		% gray
\definecolor{pystatecol}{RGB}{198,112,224}		% violet
\definecolor{pystringcol}{RGB}{176,230,134}		% green
\definecolor{pyinbuiltscol}{rgb}{0.55,0.15,0.55}	% plum
\definecolor{pyspecialcol}{rgb}{0.8,0.45,0.12}	% orange
\definecolor{spyderbackground}{RGB}{25,35,45}		% Background
\definecolor{basiccol}{RGB}{255,255,255}			% white
\definecolor{numbercol}{RGB}{250,237,92}			% number

\lstdefinestyle{stylepython}{
	    inputencoding=utf8,
        language=Python, 
        basicstyle=\color{basiccol}\pyfontfamily, 
        commentstyle=\color{pycommentcol}\itshape,
        emph={self,cls,@classmethod,@property}, % Custom highlighting
		emphstyle=\color{pyspecialcol}\itshape, % Custom highlighting style
		numberstyle=\color{numbercol}
		morestring=[b]{"""},
		stringstyle=\color{pystringcol},
		keywordstyle=\color{pystatecol},        % statements
		% remove any inbuilt functions from keywords
		deletekeywords={print},
		% add any inbuilts, not statements
		morekeywords={print,None,TypeError},
		keywordstyle=\color{pyinbuiltscol},   
        numberstyle=\tiny,  
%       mathescape,  
%       showstringspaces=false,   
%       tabsize=2,   
        framexleftmargin=5mm,  
%       framexrightmargin=5pt,
%       framexbottommargin=5pt
        xleftmargin=0mm,  
        keepspaces=false,   
        classoffset=1,     
        numbers=left,    
        stepnumber=1,    
        numbersep=8pt,   
        showstringspaces=false,  
        frame=single,
        framerule=1pt,
        rulecolor=\color{black}, 
%       breaklines=true,  
%       rulesepcolor=\color{blue}, %avec frame=shadowsbox
        backgroundcolor=\color{spyderbackground}
}

\lstset{literate=%
   *{0}{{{\color{numbercol}0}}}1
    {1}{{{\color{numbercol}1}}}1
    {2}{{{\color{numbercol}2}}}1
    {3}{{{\color{numbercol}3}}}1
    {4}{{{\color{numbercol}4}}}1
    {5}{{{\color{numbercol}5}}}1
    {6}{{{\color{numbercol}6}}}1
    {7}{{{\color{numbercol}7}}}1
    {8}{{{\color{numbercol}8}}}1
    {9}{{{\color{numbercol}9}}}1
}

\begin{document}

\begin{titlepage}
  \begin{sffamily}
  \begin{center}

    % Upper part of the page. The '~' is needed because \\
    % only works if a paragraph has started.
    \includegraphics[scale=0.15]{img1.EPS}~\\[1.5cm]

    \textsc{\LARGE INSA de Toulouse}\\[2cm]

    \textsc{\Large Compte Rendu TP d'analyse numérique 1}\\[1.5cm]

    % Titre
    \HRule \\[0.4cm]
    { \huge \bfseries Application des séries de Fourier à l’étude du violon\\[0.4cm] }

    \HRule \\[2cm]
    \includegraphics[scale=0.25]{img2.EPS}\\[4cm]

    % Auteurs
      \begin{flushleft} \large
        Ducloy Léo\\
        Oscamou Kévin\\
        Promo 57\\
      \end{flushleft}

    \vfill

    % Bas de page
    {\large 9 Janvier 2022}

  \end{center}
  \end{sffamily}
\end{titlepage}

\tableofcontents
\newpage

\section{Introduction}
\subsection{Résumé}
Nous avons eu deux types de séances de travaux pratique en analyse numérique. Des séances tournées vers l'approche théorique et d'autres vers la programation en \textit{Python}.
\subsection{Objectifs}
L’objectif de ce TP est de créer un semblant de violon numérique en faisant jouer des sons de
"violon" à l’ordinateur. Ce TP comporte des travaux théoriques de modélisation du mouvement
d’une corde de violon et des travaux numériques de programmation de ces résultats.

A l’issue des 2 séances de TD et 3 séances de TP (1 séance de TD préparatoire au premier TP,
1 séance préparatoire aux deux autres), vous rendrez par \textbf{binôme} un \textbf{rapport} sur l’ensemble
du TP (théorique + numérique). Les questions à traiter en TP (et donc à programmer) sont
indiquées par le symbole suivant \fbox{\textbf{TP}} .

\section{\'Etude du mouvement de la corde de violon}
La corde en question possède les propriétés physiques suivante : \\
\begin{itemize}
\item Une masse par unité de longueur $\mu = 6.10^{-4} \ kg.m^{-1}$,
\item Une tension $\tau=200 \ N$ réglée lors de l'accord du violon par le musicien,
\item Une longueur $L=0.5 \ m$. \\
\end{itemize}
La relation fondamentale de la dynamique permet alors d'écrire
\begin{equation}
\label{eqn:d2tftx}
	\partial^{2}_{t} f(t,x)=c^2\partial^{2}_{x} f(t,x) \quad x \in [0,L], \quad t \geq 0
\end{equation}
où $c=\sqrt{\frac{\tau}{\mu}}>0$. On lui associe les conditions aux limites et conditions initiales suivantes :
\begin{equation}
\label{eqn:conditionsftx}
	\begin{matrix*}[l]
		\ (i)	&f(t,0)=f(t,L)=0 									&\forall t \geq 0, \\
		(ii) 	&f(0,x)=0, \ \partial_{t}f(0,x)=\alpha(L-x)\qquad 	&\forall x \in [0,L],
	\end{matrix*}
\end{equation} 
où $\alpha=0.1$. \\
On pose $u=x-ct$ et $v=x+ct$ et on définit l'application linéaire $\varphi$ de la facon suivante :
\[
\begin{matrix}
\varphi:\mathbb{R}^{+}\times [0,L] & \rightarrow & \varphi(\mathbb{R}^{+}\times [0,L])=\Omega \\
(t,x) & \rightarrow & (u+v)=(x-ct,x+ct) \\
\end{matrix}
\]
De plus on pose
\begin{equation}
\label{eqn:ftx}
f(t,x)=F(x-ct,x+ct)=F(u,v)
\end{equation}
\newpage
\subsection{Résolution par onde progressive}
\subsubsection{Montrer que $\varphi$ est un difféomorphisme de classe $C^2$}
\noindent Tout d'abord, l'application $\varphi$ est un difféomorphisme de clase $C^2$
\[
ssi : \
\begin{cases}
\varphi$ est de classe $C^2 \\
\varphi$ est bijective $ \\
\varphi^{-1}$ est de classe $C^2 \\
\end{cases}
\]
\begin{itemize}
\item $\varphi$ est de classe $C^1$ ssi $\frac{\partial f}{\partial t}$ et $\frac{\partial f}{\partial x}$ existent et sont continue
\item $\varphi$ est de classe $C^2$ ssi $\frac{\partial f}{\partial t}$ et $\frac{\partial f}{\partial x}$ existent et sont de classe $C^1$ \\
\end{itemize}
\[
\forall (t,x)\in \mathbb{R}^2,\quad \frac{\partial f}{\partial t}(t,x)=(-c,c) \text{ et } \frac{\partial f}{\partial x}(t,x)=(1,1)
\]
Ce sont deux des fonctions constante, donc de classe $C^{\infty}$ \\
On a enfin, \\
\[
\forall (t,x)\in \mathbb{R}^2,\quad \frac{\partial^2 f}{\partial t^2}(t,x)=\frac{\partial^2 f}{\partial x^2}(t,x)=\frac{\partial^2 f}{\partial t \partial x}(t,x)=\frac{\partial^2 f}{\partial x \partial t}(t,x)=(0,0)
\]
$\varphi$ est de classe $C^2$ sur $\mathbb{R}^2$ \\ \\
Pour que $\varphi$ soit bijective il faut qu'il existe un unique couple $(t,x)$ appartenant à $\mathbb{R}^+ \times [0,L]$ tel que $(u,v)=\varphi (t,x)$ \\
Or on a ,
\[
\begin{matrix}
u=x-ct & \text{ et } & v=x+ct \\
t=\frac{v-u}{2c} & \text{ et } & x= \frac{u+v}{2}
\end{matrix}
\]
Alors,
\begin{equation}
\varphi^{-1}(u,v)=(\frac{v-u}{2c},\frac{u+v}{2})
\end{equation}
$\varphi$ est donc bijective sur $\mathbb{R}^2$
\[
\forall (u,v)\in \mathbb{R}^2,\quad \frac{\partial \varphi^{-1}}{\partial u}(u,v)=(\frac{-1}{2c},\frac{1}{2}) \text{ et } \frac{\partial \varphi^{-1}}{\partial v}(u,v)=(\frac{1}{2c},\frac{1}{2})
\]
Ce sont deux des fonctions constante, donc de classe $C^{\infty}$ \\
Finalement, $\varphi$ est bien un difféomorphisme de classe $C^2$
\subsubsection{Calculer $\Omega = \varphi(\mathbb{R}^{+}\times [0,L])$}
\noindent On a, $\Omega=\varphi (\mathbb{R}^+ \times [0;L])$ \\ \\
Et on sait que : $t=\frac{v-u}{2c} \geq 0 \qquad $et$ \qquad 0 \leq \frac{u+v}{2}=x \leq L$ \\ \\
Cela implique que $\forall (u,v)\in \mathbb{R}^2 \quad $ on a$ \quad v\geq u, \qquad v\geq -u, \qquad v\leq 2L-u$
\newpage
On obtient la représentation d'$\Omega$ suivante :
\begin{figure}[!h]
\centering
\begin{pspicture}(-5,-5)(5,5)
\psaxes[xLabels={,-4L,-3L,-2L,-L,0,L,2L,3L,4L},yLabels={,-4L,-3L,-2L,-L,0,L,2L,3L,4L}]{->}(0,0)(-5,-5)(5,5)[$u$,-45][$v$,180]
\psgrid[griddots=5, subgriddiv=0, gridlabels=0pt](-5,-5)(5,5)
\pspolygon[fillstyle=hlines,hatchcolor=gray,linestyle=none](-5,5)(0,0)(1,1)(-3,5)
\psline[linecolor=blue](5,5)(-5,-5)
\psline[linecolor=red](-5,5)(5,-5)
\psline[linecolor=green](-3,5)(5,-3)
\rput(4,4.75){\color{blue}$v=u$}
\rput(-4,3.25){\color{red}$v=-u$}
\rput(4.6,-1.5){\color{green}$v=2L-u$}
\rput(-2,3){\large\textbf{$\Omega$}}
\end{pspicture}
\caption{Représentation d'$\Omega$}
\end{figure}

Finalement, $\Omega=[-\infty ;L]\times \mathbb{R}^+$
\subsubsection{Exprimer $\partial^{2}_{t} f(t,x)$ en fonction de $ \partial^{2}_{u} F(u,v),\partial^{2}_{v} F(u,v),\partial^{2}_{uv} F(u,v)$}
Tout d'abord on a : $f(t,x)=F(u,v)=F(\varphi (t,x))=F\circ \varphi (t,x)$
\[
\begin{matrix}
\frac{\partial f}{\partial t}(t,x) & = & \frac{\partial F}{\partial u}(\varphi(t,x)).\frac{\partial u}{\partial t} + \frac{\partial F}{\partial v}(\varphi(t,x)).\frac{\partial v}{\partial t} \\[0.4cm]
\frac{\partial}{\partial t}F\circ \varphi(t,x) & = & -c\frac{\partial F}{\partial u}(\varphi(t,x)) +c\frac{\partial F}{\partial v}(\varphi(t,x))
\end{matrix}
\]
On a par la suite :
\begin{align}
\frac{\partial^2 f}{\partial t^2}(t,x) &=\frac{\partial}{\partial t}\left(\frac{\partial f}{\partial t}\right) \notag \\
&=\frac{\partial}{\partial t}\left( -c\frac{\partial F}{\partial u}(\varphi(t,x)) + c\frac{\partial F}{\partial v}(\varphi(t,x))\right) \notag \\ 
&=-c\frac{\partial}{\partial t}\left( \frac{\partial F}{\partial u}\circ\varphi(t,x)\right) + c\frac{\partial}{\partial t}\left( \frac{\partial F}{\partial v}\circ\varphi(t,x))\right) \notag \\
&=-c\left( -c\frac{\partial^2 F}{\partial u^2} + c\frac{\partial^2 F}{\partial u \partial v} \right) + c\left( -c\frac{\partial^2 F}{\partial u \partial v} + c\frac{\partial^2 F}{\partial v^2} \right) \notag \\
\Aboxed{\frac{\partial^2 f}{\partial t^2}(t,x) &=c^2\frac{\partial^2 F}{\partial u^2}-2c^2\frac{\partial^2 F}{\partial u \partial v}+c^2\frac{\partial^2 F}{\partial v^2}}
\end{align}
\subsubsection{Exprimer $\partial^{2}_{x} f(t,x)$ en fonction de $ \partial^{2}_{u} F(u,v),\partial^{2}_{v} F(u,v),\partial^{2}_{uv} F(u,v)$}
De la même manière on obtient $\frac{\partial^2 f}{\partial x^2}(t,x)$
\[
\begin{matrix}
\frac{\partial f}{\partial x}(t,x) & = & \frac{\partial F}{\partial u}(\varphi(t,x)).\frac{\partial u}{\partial x} + \frac{\partial F}{\partial v}(\varphi(t,x)).\frac{\partial v}{\partial x} \\[0.4cm]
\frac{\partial}{\partial x}F\circ \varphi(t,x) & = & \frac{\partial F}{\partial u}(\varphi(t,x)) +\frac{\partial F}{\partial v}(\varphi(t,x))
\end{matrix}
\]
On a par la suite :
\begin{align}
\frac{\partial^2 f}{\partial x^2}(t,x) &=\frac{\partial}{\partial x}\left(\frac{\partial f}{\partial x}\right) \notag \\
&=\frac{\partial}{\partial x}\left( \frac{\partial F}{\partial u}(\varphi(t,x)) + \frac{\partial F}{\partial v}(\varphi(t,x))\right) \notag \\ 
&=\frac{\partial}{\partial x}\left( \frac{\partial F}{\partial u}\circ\varphi(t,x)\right) + \frac{\partial}{\partial t}\left( \frac{\partial F}{\partial v}\circ\varphi(t,x))\right) \notag \\
\Aboxed{\frac{\partial^2 f}{\partial x^2}(t,x) &=\frac{\partial^2 F}{\partial u^2}+\frac{\partial^2 F}{\partial u \partial v}+\frac{\partial^2 F}{\partial v^2}}
\end{align}
\subsubsection{Si $f$ est solution de $\eqref{eqn:ftx}$, trouver l'EDP satisfaite par $F$}
D'après l'équation \eqref{eqn:conditionsftx} on peut écrire :
\begin{align}
0	&=\frac{\partial^2 f}{\partial t^2}(t,x)-c^2\frac{\partial^2 f}{\partial x^2}(t,x) \notag \\
	&=c^2\frac{\partial^2 F}{\partial u^2}-2c^2\frac{\partial^2 F}{\partial u \partial v}+c^2\frac{\partial^2 F}{\partial v^2}-c^2\frac{\partial^2 F}{\partial v^2}-2c^2\frac{\partial^2 F}{\partial u \partial v}-c^2\frac{\partial^2 F}{\partial v^2} \notag \\
	&=-4c^2\frac{\partial^2 F}{\partial u \partial v} \Longrightarrow \boxed{\frac{\partial^2 F}{\partial u \partial v}(u,v)=0} 
\end{align}
On trouve finalement que l'équation au dérivées partielles satisfaite par $F$ si $f$ est solution de \eqref{eqn:d2tftx} est : $\frac{\partial^2 F}{\partial u \partial v}(u,v)=0$
\subsubsection{Montrer que $F$ peut s'écrire $F(u,v)=h(u)+g(v)$}
\noindent Les fonctions $h$ et $g$ sont de classe $C^2$ sur $\mathbb{R}$. On a,
\begin{equation*}
	\frac{\partial}{\partial u}\left( \frac{\partial F}{\partial v} \right)=0 \Longrightarrow \frac{\partial F}{\partial v} =C
\end{equation*}
où $C$ est une constante par rapport à $u$. Alors $\frac{\partial F}{\partial v}=a(v)$. De manière équivalente on trouve que $\frac{\partial F}{\partial u}=b(u)$.
\begin{equation*}
	F(u,v)=A(v)+B(u)
\end{equation*}
\begin{equation*}
	f(t,x)=F(\varphi(t,x))=A(x+ct)+B(x-ct)\qquad \forall(t,x) \in \mathbb{R}^+ \times [0,L]
\end{equation*}
\subsubsection{Trouver $h$ et $g$ pour que $f$ soit solution de \eqref{eqn:conditionsftx}}

D'après la condition aux limites (\ref{eqn:conditionsftx}.i) on obtient $h(-ct)+g(ct)=0$. Et par linéarité on peut écrire $h(-t)=-g(t)$. Finalement, $f$ est de la forme :
\begin{equation*}
	f(t,x)=h(x-ct)-h(-x-ct)
\end{equation*}
De plus grâce à la condition initiale (\ref{eqn:conditionsftx}.ii) on obtient $h(x)-h(-x)=0 \Longleftrightarrow h(x)=h(-x)$. \textbf{h est donc paire}. On déduit de la condition aux limites (\ref{eqn:conditionsftx}.i) :
\begin{equation*}
	h(L-ct)-h(-L-ct)=0 \Longleftrightarrow h(L-ct)=h(-L-ct)
\end{equation*}
En posant $z=L-ct$ on obtient $h(2L-z)=h(z)$. Etant donné que $h$ est paire, on peut dire  que $h$ est \textbf{2L-périodique} et que $f$ s'écrit $f(t,x)=h(x-ct)-h(x+ct)$. De plus,
\begin{align*}
	\partial_{t}f(t,x)	&=-c.h'(x-ct)-c.h'(x+ct) \\
	\partial_{t}f(0,x)	&=-c.h'(x)-c.h'(x)=-2c.h'(x)
\end{align*}
En utilisant maintenant la deuxième condition initiale (\ref{eqn:conditionsftx}.ii) on peut écrire $\partial_{t}f(0,x)=-2c.h'(x)=\alpha(L-x)$. Alors $\forall x \in [0,L]$ on a
\begin{align*}
	h'(x)	&=\frac{-\alpha}{2c}(L-x) \\
	h(x)		&=\frac{\alpha}{4c}(L-x)^2+C
\end{align*}
Or, $f(t,L)=0 \Longrightarrow h(L)=0 \Longrightarrow C=0$ donc $h(x)=\frac{\alpha}{4c}(x-L)^2$ de plus pour assurer la partié de $h$ on absolutise la variable. On obtient alors $\forall x \in [-L,L]$
\begin{equation*}
	h(x)=\frac{\alpha}{4c}(L- \lvert x \rvert )^2
\end{equation*}
Etant donner que $g(x)=-h(-x)$ et que $f(t,x)=h(x-ct)+g(x+ct)$
\begin{equation*}
f(t,x)=\frac{\alpha}{4c}\left( \left( L-\lvert x-ct \rvert)\right)^2-\left(L-\lvert x+ct \rvert \right)^2\right)
\end{equation*}
Avec h et g de la forme suivante
\begin{align}
	\Aboxed{h(x-ct) &= \frac{\alpha}{4c}\left(L-\lvert x-ct \rvert \right)^2} \\
	\Aboxed{g(x+ct) &= -\frac{\alpha}{4c}\left(L-\lvert x+ct \rvert \right)^2}
\end{align}
On se propose maintenant de déterminer l'intervalle de temps sur lequel $f$ décrit le mouvement de la corde et donc de déterminer également la période $T$.
\begin{alignat*}{4}
	-L 			&\leq && \ x-ct \				&&\leq L \\
	-L-x 		&\leq && \centermathcell{-ct}	&&\leq L-x \\
	-2L 			&\leq && \centermathcell{-ct}	&&\leq L \\
	\frac{-L}{c}	&\leq && \centermathcell{t}		&&\leq \frac{2L}{c} \\
	\frac{-T}{2}	&\leq && \centermathcell{t}		&&\leq T
\end{alignat*}
De la même manière pour $x+ct \in [-L,L]$ on obtient $-T \leq t \leq \frac{T}{2}$. On obtient finalement que $\frac{-T}{2} \leq t \leq \frac{T}{2}$ avec $T=\frac{2L}{c}$.
\newpage
\subsubsection{Tracage des fonctions $h$, $g$ et $f$ à l'aide de \textit{python}}
Pour tracer les courbes de $h$, $g$ et $f$ pour des valeur de $t=0,\frac{T}{4},\frac{T}{2},\frac{3T}{4}$ et $T$ on utiliseras le langage de programmation \textit{Python}. Voici ci-dessous le code des 3 fonctions :

\begin{figure}[h]
\centering
\begin{minipage}{12cm} %--------------------------------------
\begin{lstlisting}[style=stylepython]
#Definition de nos fonction g, h et f
def solexacte_g(t,x):
    tp = (t+T/2)%T-T/2
    sg = -(alpha/(4*c))*((L-abs(x+c*tp))**2)
    return sg

def solexacte_h(t,x):
    tp = (t+T/2)%T-T/2
    sh = (alpha/(4*c))*((L-abs(x-c*tp))**2)
    return sh

def solexacte(t,x):
    tp = (t+T/2)%T-T/2
    s = (alpha/(4*c))*((L-abs(x-c*tp))**2)
    - (alpha/(4*c))*((L-abs(x+c*tp))**2)
    return s

\end{lstlisting}
\end{minipage} %--------------------------------------------
\caption{Code des 3 fonction $h$, $g$ et $f$}
\end{figure}
Après avoir explicité toutes nos variables on peut réaliser le tracage des courbes pour différentes valeurs de $T$. \\

\begin{figure}[!h]
\centering
\includegraphics[scale=1]{Courbes1.EPS}\\
\caption{Courbes de $g$, $h$ et $f$ pour différentes valeurs de $t$}
\end{figure}

Sur ces quatres courbes, on peut voir l'approximation de la corde grâce à $f$ en noir. La courbe est dans les quatres cas fixée aux extrémités en $x=0$ et $x=L$.De plus, elle semble avoir un comportement cohérant avec la réalité avec un mouvement oscillant et étant T-périodique comme on peut le voir ci-dessous avec la courbe en $t=T$. En effet, la corde à la même position qu'en $t=0$. Cependant, on remarque que c'est également le cas en $\frac{T}{2}$ mais ici on se trouve à la demi période.

\begin{figure}[!h]
\centering
\includegraphics[scale=1]{Courbes1aT.EPS}\\
\caption{Courbes de $g$, $h$ et $f$ pour $t=T$}
\end{figure}

\noindent Voici ci-après, une animation des trois courbes sur toute la période $T$. \\
\url{https://urlz.fr/h26l}
\subsection{Résolution par méthode de séparation des variables}
\subsubsection{Recherche d'une solution élémentaire $F_n(t,x)$ avec $n \in \mathbb{R}^*$}
Tout d'abord nous avons le problème $(E_1)$ suivant :	
\begin{equation}
\label{eqn:probleme}
(E_{1}) \
\begin{cases}
\partial^{2}_{t} f(t,x)=c^2\partial^{2}_{x} f(t,x) &\forall x \in [0,L], \forall t \geq 0 \\
f(t,0)=u(t,L)=0, &\forall t \leq 0 \\
f(0,x)=0 \quad \partial_t f(0,x)=\alpha (L-x) \quad &\forall x \in [0,L]
\end{cases}
\end{equation}
On cherche une fonction $f(t,x)$ de la forme
\begin{equation}
\label{eqn:Eq1sepvar}
	f(t,x)=\varphi(x) \psi(t), \quad \forall(t,x) \in \mathbb{R}^+ \times [0,L]
\end{equation}
vérifiant les conditions aux limites (\ref{eqn:conditionsftx}.i). \\
On peut déterminer les expressions des dérivées partielles de $f(t,x)$
\begin{alignat*}{2}
	&\partial_{t} f(t,x)		&&=\varphi(x) \psi'(t) \\
	&\partial^{2}_{t} f(t,x)	&&=\varphi(x) \psi''(t) \\
	&\partial_{x} f(t,x)		&&=\varphi'(x) \psi(t) \\
	&\partial^{2}_{x} f(t,x)	&&=\varphi''(x) \psi(t)
\end{alignat*}
En remplaçant l'expression\eqref{eqn:Eq1sepvar} de $f$ dans l'équation \eqref{eqn:d2tftx} avec les expressions de $\partial^{2}_{t} f(t,x)$ et $\partial^{2}_{x} f(t,x)$
\begin{equation*}
	\varphi(x) \psi''(t)=c^2.\varphi''(x) \psi(t)
\end{equation*}
On obtient alors le système d'équation suivant :
\begin{equation*}
	\exists \lambda \in \mathbb{R} \ \text{tel que} \
	\begin{cases}
		\psi ''(t)-\lambda \psi (t) =0 \\
		c^2 \varphi''(x)-\lambda \varphi (x)=0	
	\end{cases}
\end{equation*}
En s'intéressant aux conditions aux limites (\ref{eqn:conditionsftx}.ii) on a
\begin{equation*}
	\begin{cases}
		f(t,0)=\varphi(0) \psi(t)=0 \\
		f(t,L)=\varphi(L) \psi(t)=0	
	\end{cases}
\end{equation*}
Cependant,on cherche un solution non nulle, ce qui implique $\varphi(0)=\varphi(L)=0$

\noindent $\bullet$ Résolution de l'équation différentielle ordinaire en $x$:
\begin{equation*}
	(E_2) \
	\begin{cases}
		c^2 \varphi''(x)-\lambda \varphi (x)=0 \qquad \forall x \in [0,L] \\
		\varphi(0)=\varphi(L)=0
	\end{cases}
\end{equation*}
Polynôme caractéristique : $c^2 r^2-\lambda = 0 \Longrightarrow r^2 =\frac{\lambda}{c^2}$ \\

\noindent $\ast$ \underline{Cas 1 : $\lambda>0$} : \\
$r=\pm \sqrt{\frac{\lambda}{c^2}} \quad \Longrightarrow \varphi(x)=Ae^{\sqrt{\frac{\lambda}{c^2}}x}+Be^{-\sqrt{\frac{\lambda}{c^2}}x}$ \\
Avec les conditions aux limites on a :
\begin{equation*}
	\begin{cases}
		\varphi(0)=A+B=0 &\Longrightarrow A=-B \\
		\varphi(L)=Ae^{\sqrt{\frac{\lambda}{c^2}}L}+Be^{-\sqrt{\frac{\lambda}{c^2}}L} &\Longrightarrow A=B=0
	\end{cases}
\end{equation*}
Finalement, $\varphi(x)=0$ donc pas de solution non nulle. \\

\noindent $\ast$ \underline{Cas 2 : $\lambda=0$} : \\
$r=0 \quad \Longrightarrow \varphi(x)=Ax+B$ \\
Avec les conditions aux limites on a :
\begin{equation*}
	\begin{cases}
		\varphi(0)=B=0 &\Longrightarrow A=-B \\
		\varphi(L)=AL=0 &\Longrightarrow A=0
	\end{cases}
\end{equation*}
Finalement, $\varphi(x)=0$ donc pas de solution non nulle. \\

\noindent $\ast$ \underline{Cas 3 : $\lambda<0$} : \\
$r=\pm i\sqrt{\frac{\lambda}{c^2}} \quad \Longrightarrow \varphi(x)=A\ cos\left(\frac{\sqrt{-\lambda} x}{c}\right) +B\ sin\left(\frac{\sqrt{-\lambda} x}{c}\right)$\\
Avec les conditions aux limites on a :
\begin{equation*}
	\begin{cases}
		\varphi(0)=A=0  \\
		\varphi(L)=B\ sin\left(\frac{\sqrt{-\lambda} L}{c}\right)=0
	\end{cases}
\end{equation*}
Pour avoir une solution non nulle, il faut que $B \neq 0$ et donc que $sin\left(\frac{\sqrt{-\lambda} L}{c}\right)=0$ \\
Soit $\frac{\sqrt{\lambda}L}{c}=n\pi$ avec $n \in \mathbb{N} \Longrightarrow \lambda_n=-\left(\frac{cn\pi}{L}\right)^2$ \\
Finalement, $\boxed{\varphi_n(x)= sin \left(\frac{cn\pi}{L}\right)^2=sin \left(k_n x)\right)}$ avec $k_n=\frac{n\pi}{L}$ \\

\noindent $\bullet$ Le produit scalaire usuel qui ortogonalise la famille de fonction $\varphi_n$ s'écrit
\begin{equation*}
\left\langle \varphi_n \vert \varphi_m \right\rangle = \int_{0}^{L} sin \left(\frac{n\pi x}{L}\right) sin \left(\frac{m\pi x}{L}\right)
\end{equation*}
De plus $\left\langle \varphi_n \vert \varphi_m \right\rangle=\frac{L}{2}$ si $n=m, \quad 0$ sinon \\

\noindent $\bullet$ Résolution de l'équation différentielle ordinaire en $t$ :
\begin{equation}
	(E_3) \ \psi_n''(t)-\lambda_n \psi_n(t)=0
\end{equation}
Polynôme caractéristique : $r^2-\lambda_n =0 \Longrightarrow r^2=\lambda_n$ \\
Or, on a $\displaystyle \lambda_n<0 \qquad r=\pm i\sqrt{-\lambda_n}=\pm i \frac{n\pi c}{L}$\\
Alors $\psi_n$ est de la forme $\displaystyle \boxed{\psi_n(t)= A_n \, cos \left(\frac{cn\pi}{L}t\right)+B_n \, sin \left(\frac{cn\pi}{L}t\right)}$ \\[0.2cm]
Avec $A_n, B_n \in \mathbb{R}$ et $f_n=\frac{nc}{2L}$ on a $psi_n=A_n \, cos \left(2\pi f_n t\right)+B_n \, sin \left(2\pi f_n t\right)$ \\
Finalement, une solution élémentaire du problème s'écrit $F_n(t,x)=\psi(t)\varphi(x)$ donc
\begin{equation*}
	F_n(t,x)=\left( A_n \, cos \left(2\pi f_n t\right)+B_n \, sin \left(2\pi f_n t\right)\right)\,sin(k_n x)
\end{equation*}
\subsubsection{Lien entre $f_1$ et $f_n$}
On a $f_1=\frac{c}{2L}$ et $f_n=\frac{nc}{2L}$ donc $\boxed{f_n=nf_1}$ \\
\subsubsection{Recherche de la solution générale de l'équation des ondes}
Soit $f(t,x)=\sum\limits_{n\geq 1} F(t,x)=\sum\limits_{n\geq 1} \psi(t)\varphi(x)$ donc
\begin{equation}
	\boxed{f(t,x)=\sum\limits_{n\geq 1}\left( A_n \, cos \left(2\pi f_n t\right)+B_n \, sin \left(2\pi f_n t\right)\right)\,sin(k_n x)}
\end{equation}
\subsubsection{Déterminer les constantes $A_n$ et $B_n$}
En utilisant les conditions initiales (\ref{eqn:conditionsftx}.ii) on a $f(0,x)=0=\sum\limits_{n=1}^{+\infty} An\,sin(k_n x)$ \\
Par unicité des coefficients dans le développement en série de Fourier $A_n=0$ \\
De plus on a $\partial_t f(0,x)=\alpha(L-x)=\sum\limits_{n=1}^{+\infty} 2\pi f_n B_n sin(k_n x)$ \\
Avec $\sum\limits_{n=1}^{+\infty} 2\pi f_n B_n sin(k_n x)$ 2L-périodique impaire.
\newpage
On définie un fonciton $u(x)=\alpha(L-x)$ 2L-périodique impaire  \\
\begin{alignat*}{2}
	a_n	&=\frac{2}{2L}	\int_{-L}^{L} u(x)\, cos(k_n x)\, dx=0 \\
	b_n	&=\frac{2}{2L}	\int_{-L}^{L} u(x)\, sin(k_n x)\, dx \\
		&=\frac{2}{L}	\int_{0}^{L} u(x)\, sin(k_n x)\, dx \\
		&=\frac{2}{L}	\int_{0}^{L} \alpha(L-x)\, sin(k_n x)\, dx \\
	b_n	&=\frac{2}{k_n}
\end{alignat*}
On a alors
\begin{alignat*}{2}
	\alpha(L-x)	&=\sum\limits_{n=1}^{+\infty} 2\pi f_n B_n sin(k_n x) \\
				&=\sum\limits_{n=1}^{+\infty} \frac{2\alpha}{k_n} sin(k_n x)
\end{alignat*}
Donc $\displaystyle B_n=\frac{\alpha}{\pi f_n k_n}=\frac{2L^2 \alpha}{cn^2\pi^2}$ \\
En résumé $\boxed{A_n=0}$ et $\displaystyle \boxed{B_n=\frac{2L^2 \alpha}{cn^2\pi^2}}$
\subsubsection{Expression de l'unique solution de \eqref{eqn:d2tftx} et \eqref{eqn:conditionsftx}}
L'unique expression de la solution est donc :
\begin{equation}
	f(t,x)=\sum\limits_{n=1}^{+\infty} \frac{2L^2 \alpha}{cn^2\pi^2}\,sin(2\pi f_n t)\,sin(k_n x)
\end{equation}
\subsubsection{Démontrer que l'érreur est inférieure à $\frac{2\alpha L}{c\pi^2 p}$}
On a 
\begin{alignat*}{3}
	u_p(t,x)&=\sum\limits_{n=1}^{p} \frac{2L^2 \alpha}{cn^2\pi^2}\,sin(k_n x)\,sin(2\pi f_n t)\quad &&\forall(t,x) \\
	\lvert f(t,x)-u_p(t,x) \rvert &=\sum\limits_{n=p+1}^{+\infty} \frac{2L^2 \alpha}{cn^2\pi^2}\,\lvert sin(k_n x)\,sin(2\pi f_n t) \rvert \qquad &&\forall(t,x)
\end{alignat*}
Cependant $\lvert sin(k_n x)\,sin(2\pi f_n t) \rvert \leq 1$ \\
\begin{alignat*}{2}
	\lvert f(t,x)-u_p(t,x) \rvert 	&\leq \sum\limits_{n=p+1}^{+\infty} \frac{2L^2 \alpha}{cn^2\pi^2} \\
									&\leq \frac{2L^2 \alpha}{c\pi^2} \sum\limits_{n=p+1}^{+\infty} \frac{1}{n^2} \\
									&\leq \frac{2L^2 \alpha}{c\pi^2} \int\limits_{p+1}^{+\infty} \frac{1}{n^2} dn \qquad \text{(approximation intégrale)} \\
									&\leq \frac{2L^2 \alpha}{c\pi^2} \times \frac{1}{p+1} \quad \text{or} \quad \frac{1}{p}>\frac{1}{p+1}
\end{alignat*}
On a donc $\boxed{\lvert f(t,x)-u_p(t,x) \rvert \leq \frac{2L^2 \alpha}{c\pi^2 p}}$
\subsubsection{Création d'une fonction $solapprochee(t,x,p)$ qui calcule $u_p(t,x)$ en \textit{python}}
Voici ci-dessous le code de la fonction $solapprochee$ :
\begin{figure}[h]
\centering
\begin{minipage}{12cm} %--------------------------------------
\begin{lstlisting}[style=stylepython]
def solapprochee(t,x,p):
    Up = 0
    for i in range(1, p+1):
        fn = (i*c)/(2*L)
        kn = (i*np.pi)/L
        U = (2*alpha*L*L)/(i*i*np.pi**2*c)
        *np.sin(kn*x)*np.sin(2*np.pi*fn*t)
        Up = Up + U
    return Up

\end{lstlisting}
\end{minipage} %--------------------------------------------
\caption{Code de la fonction $solapprochee$}
\end{figure}

\subsubsection{Traçage de la fonction $u_p(t,x)$ pour $p=1,3,10,100$}
Voici ci-après un graphique avec différentes courbes de $solapprochee$ pour différente valeurs de $p$ à $t=T/3$
\newpage
\begin{figure}[!h]
\centering
\includegraphics[scale=1]{Courbes3.EPS}\\
\caption{Courbes de $u_p(t,x)$ pour $p=1,3,10,100$}
\end{figure}

On remarque que plus $p$ augmente, plus l'approximation de $u_p$ est bonne. Ce qui est cohérant avec l'expression de la solution approchée.
\subsubsection{Traçage de l'érreur en fonction de $p$}
On se propose donc de tracer l'évolution de l'érreur en fonction de $p$.
\begin{figure}[!h]
\centering
\includegraphics[scale=1]{Courbes4Erreur.EPS}\\
\caption{Courbes de l'érreur pour avec la majoration}
\end{figure}

On remarque sur cette figure que l'érreur est évidemment toujours inférieure à la majoration donnée par $\displaystyle \frac{2\alpha L}{c\pi^2 p}$.
\newpage
\section{Son produit par l'instrument}
Le son produit par le violon peut être approché par la force exercée sur le chevalet par la
corde. Ce chevalet est relié à la table des harmoniques d’où la note est émise, et correspond
à la frontière de notre modèle située en $L$. Cette force est assimilée à la dérivée spatiale du mouvement de la corde en $L$. On définit donc $s(t)$ le son émis par le violon au court du temps par :
\begin{equation}
	s(t)=\frac{\partial}{\partial x}f(t,x)\vert_{x=L}=\sum\limits_{n=1}^{+\infty} d_n\, sin(2\pi f_n t)
\end{equation}
Et on définira $v_p$ une approximation du son émis par l'instrument à un instant $t$, la quantité :
\begin{equation*}
	v_p(t)=\sum\limits_{n=1}^{p} d_n\, sin(2\pi f_n t)
\end{equation*}
\subsubsection{Calcule de $d_n$}
Dans un premier temps calculons $\frac{\partial}{\partial x}f(t,x)\vert_{x=L}$ \\
\begin{alignat*}{2}
	f(t,x)										&= \sum\limits_{n=1}^{+\infty} \frac{2L^2 \alpha}{cn^2\pi^2}\,sin(2\pi f_n t)\,sin(k_n x) \\
	\frac{\partial}{\partial x}f(t,x)			&= \sum\limits_{n=1}^{+\infty} \frac{2L \alpha}{cn\pi}\,sin(2\pi f_n t)\,cos(\frac{n\pi}{L} x) \\
	\frac{\partial}{\partial x}f(t,x)\vert_{x=L}	&= \sum\limits_{n=1}^{+\infty} \frac{2L \alpha}{cn\pi}\,(-1)^n\, sin\,(2\pi f_n t)
\end{alignat*}
On trouve finalement par identification que $d_n=\frac{2\alpha L}{cn\pi}(-1)^n$

\subsubsection{Vérification que $\lvert d_n \rvert \underset{+\infty}{\backsim} \frac{1}{n}$}
On a alors $\lvert d_n \rvert = \frac{2\alpha L}{cn\pi}= \frac{2\alpha L}{c\pi}\times \frac{1}{n}$. Donc $\lvert d_n \rvert \underset{n\rightarrow +\infty}{\backsim} \frac{1}{n}$
\subsubsection{Traçage d'un diagramme en bar pour les 30 premières valeurs de $d_n$}
Voici ci-dessous le code de la fonction $D_n$ :
\begin{figure}[h]
\centering
\begin{minipage}{12cm} %--------------------------------------
\begin{lstlisting}[style=stylepython]
def Dn(N):
    dn = (2*alpha*L)/(N*np.pi*c)*np.cos(N*np.pi)
    return dn

\end{lstlisting}
\end{minipage} %--------------------------------------------
\caption{Code de la fonction $D_n$}
\end{figure}
\newpage
\noindent De plus, voici ci-dessous le diagramme en bar représentant les différentes valeurs de $d_n$ :
\begin{figure}[!h]
\centering
\includegraphics[scale=0.8]{Diagramme1.EPS}
\caption{Diagramme en bar des valeurs de $d_n$}
\end{figure}
\subsubsection{Traçage de $v_p$ sur l'intervalle $[0,T]$}
On va vouloir tracer $v_p$ sur l'intervalle $[0,T]$. Dans un premier temps, on défini $v_p$ sur python par la fonction suivante : \\
\begin{figure}[h]
\centering
\begin{minipage}{12cm} %--------------------------------------
\begin{lstlisting}[style=stylepython]
def Vp(t,P):
    vp = 0
    for n in range(1,P+1):
        vp = 2*alpha*L/(n*np.pi*c)*(-1)**n
        *np.sin((n*np.pi*c)/L*t) + vp
    return vp

\end{lstlisting}
\end{minipage} %--------------------------------------------
\caption{Code de la fonction $Vp$}
\end{figure} \\
On trace ensuite $v_p$ sur la période $T$ autrement dit, sur l'intervalle $[0,T]$. On obtient donc la figure ci-dessous : \\
\begin{figure}[h]
\centering
\includegraphics[scale=0.8]{Courbe5VpT.EPS}\\
\caption{Courbe de $v_p$ sur $T$}
\end{figure}
\subsubsection{Traçage de $v_p$ sur 2 secondes}
Après avoir visualiser $v_p$ sur une période, on se propose de visualisé le son sur une durée de 2 secondes. Pour ce faire, on trace la courbe avec la même fonction $Vp$ mais sur une durée de 2 secondes.
\begin{figure}[h]
\centering
\includegraphics[scale=1]{Courbe6Vp2sec.EPS}\\
\caption{Courbe de $v_p$ sur 2 secondes}
\end{figure}
\subsubsection{Construction d'une fonction attenuation}
Par la suite on souhaite attenuer notre fonction pour que le son produit soit plus réaliste. On défini donc une fonction enveloppe ("$envelop$") qui attenuera le début ainsi que la fin de notre son.
\begin{figure}[h]
\centering
\begin{minipage}{13cm} %--------------------------------------
\begin{lstlisting}[style=stylepython]
def envelop(Interval,N):
    env = np.ones(N)
    
    #profil de montee
    x=np.linspace(0,0.5,int(sorted(Interval)[0]*N))
    for k in range(0,len(x)-1):
        env[k]=16*((1-x[k])*x[k])**2
        *(x[k]/(2*l1))
    				
    #profil de descente
    x=np.linspace(0.5,1,int(N-sorted(Interval)[1]*N))
    for k in range(N+1-len(x),N+1):
        env[k-1]=16*((1-x[k-(N+1-len(x))])
        *x[k-(N+1-len(x))])**2
        *((1-x[k-(N+1-len(x))])/(2*(1-l2)))
    return env

attenuation = envelop([l1,l2],2*fEch+1)

\end{lstlisting}
\end{minipage} %--------------------------------------------
\caption{Code de la fonction $solapprochee$}
\end{figure}
\newpage
\subsubsection{Traçage de $v_p$ atténuée sur 2 secondes}
Après avoir défini notre fonction $envelop$ et créer notre vecteur $attenuation$ représentant le profil de montée et de descente on se propose de simplement visualiser la courbe d'attenuation.
\begin{figure}[h]
\centering
\includegraphics[scale=1]{Courbe7enveloppe.EPS}\\
\caption{Courbe de $attenuation$ sur 2 secondes}
\end{figure} \\
Maintenant, en multipliant notre fonction $Vp$ avec le vecteur $attenuation$ on obtient notre  signal atténué ci-dessous.
\begin{figure}[h]
\centering
\includegraphics[scale=1]{Courbe8SonAmorti.EPS}\\
\caption{Courbe de $v_p*attenuation$ sur 2 secondes}
\end{figure}
\subsubsection{Création des sons}
Après avoir correctement défini nos fonctions du son et du son atténué. On se propose de générer les sons à l'aide d'un module python. Disponible en exécutant le programme python ou en lien ci-après.
\section{Annexe}
Lien du programme python : \href{https://github.com/DucloyLeo/TPAN2021.git}{Cliquez-ici} \\

Lien téléchargement du son non amorti sur 2 secondes : \href{https://drive.google.com/uc?export=download&id=19MVfgDUoqnFzE0GcLlgDqclY-TrsQStx}{Cliquez-ici} \\

Lien téléchargement du son amorti sur 2 secondes : \href{https://drive.google.com/uc?export=download&id=1E1k3nFMNvk4p9HQP9ZI2LJBZikAHHUWY}{Cliquez-ici} \\

Lien du programme LaTeX : \href{https://github.com/DucloyLeo/TPAN2021.git}{Cliquez-ici} \\
\end{document}
